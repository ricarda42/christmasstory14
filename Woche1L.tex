Ich habe seit Wochen kaum geschlafen, ich will es endlich vollbringen. Auf keinen Fall will ich Weihnachten allein sein. Diese Jahreszeit ist einfach nicht für Einzelgänger gemacht — es ist als würde einem die Welt zu rufen: „Sieh dich doch um, sieh endlich ein: Du bist allein, du bleibst allein! Allein, allein, allein, allein!“ Aber bald ist es vorbei. Bald bin ich nicht mehr allein. Dann bin ich auf niemanden mehr angewiesen, dann habe ich es als erster geschafft — ich werde der erste Mensch sein der sich aus dem Abgrund der sozialen Abhängigkeit selbst befreit. Mich kann keiner mehr aufhalten! Und dann bin ich nie wieder allein!

Ding Dong. Keine Antwort. Ding Dong. ‚Mama, lass mich rein, ich will meine Wäsche abholen.“ Warum braucht die Alte immer so lange? Die Tür geht auf und meine Mutter überspringt gleich die Begrüßung: „Warum kannst deine Wäsche eigentlich nicht selbst machen, wie alle anderen? Du bist 33!“ Ich entgegne ihr: „Ich habe nun wirklich nicht die Zeit mich mit derlei trivialen Problemen herumzuschlagen… Aber was weißt du schon.“ Der letzte Satz war erwartungsgemäß nicht wirklich zielführend, er führte unausweichlich zu einer elend langen Predigt, der ich mich nur entwinden konnte mit der Ausrede ich würde mich noch mit Cindy treffen. Cindy hat mich natürlich schon längst abblitzen lassen, ich wäre zu selbstbezogen, narzisstisch — egal, ich bin ohne sie ehso besser dran. Wer mich nicht schätzen kann, ist halt nicht gut genug für mich. Aber meine Mutter scheint die Vorstellung, es könnte eine Frau in meinem Leben geben zu beruhigen, und solange das hilft, dass sie macht was ich will, soll es mir recht sein. Bald werde ich Cindy ersetzen und das Ergebnis wird besser als alle Cindys dieser Welt! Auf dem Weg nach Hause fällt mir noch irgendwas auf%nitrneatuirn

Es gibt immer noch Tests der KI (künstliche Intelligenz, Anm. d. A.), die fehlschlagen. Das freie Denken ermöglicht de KI zu viel Ungehorsam. Aber er muss morgen fertig werden! Und natürlich soll er mir gehorchen, ich will schließlich nicht ein weiteres Problem erschaffen. Ich muss wohl einen Kompromiss zwischen freiem Denken und Fügsamkeit finden, zwischen Freiheit und Sicherheit. Das ist aber schwierig — ich fühle mich wie eine nicht-triviale Indexmenge! % unbedingt Fußnote zur erklärung

Im Laufe der Nacht habe ich einen genialen Balanceakt zwischen den beiden Extremen entwickelt — im Gegensatz zu gewissen Regierungen. Jetzt kann ich B2-QP endlich zum Leben erwecken, ich fühle mich ein wenig wie Frankenstein! Oh, ich habe jetzt die Tests nicht gar nicht mehr laufen lassen; egal! Ich werde schon keine Fehler gemacht haben!

