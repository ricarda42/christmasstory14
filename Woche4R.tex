Ich hab mich ein bisschen eingerichtet im Geschenkelager. Ein sehr primitiver Roboter in Form eines Hundes leistet mir Gesellschaft und ein Radio habe ich auch gefunden, aus dem die neuesten Nachrichten erklingen, im Wechsel mit immer den gleichen Weihnachts-Hits — "`Last Christmas"'  habe ich jetzt schon dreizehn Mal gehört. Gerade sind wieder die Nachrichten dran: "`Das auswärtige Amt Deutschlands meldet, dass ein Bundesbürger im deutschen Staatsgebiet von der CIA festgenommen wurde. Deutschland droht den Vereinigten Staaten mit Konsequenzen, da dieses Vorgehen ungesetzlich ist. Verhaftungen auf deutschem Boden müssen von deutschen Organen ausgeführt werden. Erst dann können die Vereinigten Staaten ein Auslieferungsgesuch stellen. Das Außenministerium hat bereits den US-amerikanischen Botschafter einbestellt. Der junge Mann wurde in Lübeck in seiner Wohnung festgenommen. Er wird beschuldigt, sich in die Datenbanken der CIA eingehackt zu haben. Des weiteren wurden zahlreicher Rechner und Festplatten beschlagnahmt. Deren Durchsuchung hat ergeben, dass der 33-Jährige außerdem wenige Tage zuvor bereits einen Hacker-Angriff auf ein Londoner Wachunternehmen ausgeführt hatte. Auffällig viele der Video-Dateien, die dabei erbeutet wurden, zeigen einen erstaunlich menschlich wirkenden Roboter."'

Moment mal. Dieser Roboter, das bin ich! Und dann ist der junge Mann — Jona! Was jetzt? Offensichtlich hat er nach mir gesucht und jetzt hat ihn die CIA. Ich verstehe genug von den Gewohnheiten der US-Amerikaner, um zu wissen, dass das nichts Gutes heißen kann. Und wenn die internationales Recht brechen, nur um Jona in Haft nehmen zu können, dann kann er eigentlich nur in Askaban sein. Das alles nur, weil ich weggelaufen bin! Also gut, ich muss ihn retten, so viel ist klar. Bloß bin ja auch gefangen. Aber jetzt kommt mir zu Gute, was ich hier alles entdeckt habe. Zunächst laufe ich zu dem Stapel mit Holzspielzeug und fische ein gigantisches Pseudo-mittelalterliches Schild heraus. Dann die Ecke mit dem ganzen Science Fiction-Kram — denn hier lagert ein Laserschwert. Ob mir das was bringen wird, weiß ich zwar noch nicht, aber ich nehme es mal mit. Zuletzt der Werkzeugkasten, der versteckt hinter dem Playmobilflughaufen rumsteht. Darin finde ich zum Glück filigrane Schraubenzieher, mit denen es mir nach einigen Anläufen gelingt, die Tür der Lagerhalle zu öffnen. Mit Schild und Schwert laufe ich hinüber zu der Hütte des Weihnachtsmannes. Ich halte mich gar nicht damit auf, anzuklopfen, sondern trete einfach ein. Der Weihnachtsmann sitzt bequem in seinem Sessel und schaut mich verdutzt an. "`Was machst du denn hier?"'

"`Ich muss Jona retten! Er hat mich gesucht, dabei selbstmörderische Hacker-Angriffe gestartet und jetzt hat ihn die CIA! Du wirst mich hier nicht mehr festhalten"' — ich hebe das Laserschwert ein bisschen, in der Hoffnung, dass es bedrohlich wirkt. "`Für diesen Luke musst du dir wohl ein neues Geschenk suchen."'

"`Ach das ist eh ne verwöhnte Nervensäge. Tut dem mal ganz gut, wenn er nichts kriegt. Wenn ich es mir recht überlege — eigentlich gilt das für all die Bälger, die ich zu versorgen habe."' Einen Moment schweigt er nachdenklich. "`Pass auf, ich komme mit dir. Dieses Jahr verteile ich keine Geschenke. Los, lass uns deinen Kumpel retten!"' Mit diesen Worten schwingt er sich aus seinem Sessel, packt seine größte Rute und marschiert hinaus in den Schnee. Ich folge ihm sofort. Wir besteigen wieder den Rentier-Schlitten und schon Sekunden später jagen wir in einem rekordverdächtigen Tempo durch die Luft. "`Sag mal, wo müssen wir eigentlich hin?"'

"`Ich fürchte, sie haben nach Askaban gesteckt, das neue Guantanamo"', antworte ich niedergeschlagen.

"`Kopf hoch, wir holen deinen Freund da schon raus. Du musst es dir nur ganz fest wünschen. Wenn nicht heute ein Tag für große Wünsche ist, ich wüsste nicht, wann sonst."' Er lächelt mir zu, ein viel besseres Lächeln als ich es bei Jona je gesehen habe. Allmählich wird mir der Weihnachtsmann sympathisch. Auch wenn ich seinen plötzlichen Sinneswandel nicht ganz verstehe.

"`Wie meinst du das? Ist heute etwa schon..."'

"`Ja, heute ist Heiligabend. Bald werden sie überall feiern rund um den Globus. Und wir werden dann auch feiern, da mach dir mal keine Sorgen."'