Schon seit drei Stunden sitzt Jona und an seinem Schreibtisch und grübelt. Tippt brutal auf seine Tastatur ein und starrt dann wieder die Wand an. Dabei war ich so sicher, dass der Tag heute besser werden würde als gestern - da hat Jona kein Wort mit mir geredet. Ich bin im ganzen Haus auf und abgelaufen, hab mir jede Ecke und jeden Winkel genau angeguckt, bis mir auch das zu dumm wurde. Dann habe ich zwei mit sich selbst multipliziert, bis mein Speicher überlastet war. Irgendwann schließlich bin ich zurück in das Zimmer geschlurft, indem ich gebaut wurde und hab vor mich hingestarrt. Es war sterbenslangweilig. Heute immerhin der Ansatz eines Gesprächs und eine direkte Frage an mich! Obwohl wir an seiner Gesprächskompetenz echt noch arbeiten müssen. Doch dann ist er wieder in Grübeleien verfallen und hat nur noch durch mich hindurch geguckt. Kurze Zeit später ist er zum Schreibtisch gerannt und seitdem nicht wieder aufgestanden. Das ist nun wirklich unfair. Wieso hat er mich gebaut und mich von da an meiner Langeweile überlassen? Ich unternehme noch einen letzten Versuch. "`Jona? Hättest du nicht vielleicht Lust, spazieren zu gehen? Es hat geschneit letzte Nacht und sieht wunderschön aus draußen! Wir könnten auch einen Schneemann bauen! Oder eine Schneeballschlacht machen."' Doch Jona dreht sich nicht einmal um. "`Nein, ich geh nicht raus mit dir, und ich will erst recht keinen Schneemann bauen. Siehst du nicht, dass ich beschäftigt bin?"'"`Dann gib mir wenigstens eine Aufgabe, sag mir irgendwas, lass mich etwas tun! Es ist schrecklich langweilig, nichts zu tun zu haben und das auch noch zu wissen."' "`Jetzt hör auf, du nervst mich! Ich hab dich gebaut und ich darf über dich bestimmen! Wenn ich will, dass du still in der Ecke stehst, dann hast du das gefälligst zu tun und nicht zu hinterfragen!"' Er hat sich zu mir umgedreht, ist aufgesprungen und funkelt mich wütend an. "`Ja, du hast mich gebaut - ich frage mich nur, warum. Du willst meine Gesellschaft nicht und du lässt mich nicht für dich arbeiten. Deine Programme darf ich mir nicht einmal ansehen! Was sollte das alles?"' Jona sinkt zurück auf seinen Stuhl. "`Ich wollte nicht mehr einsam sein"', sagt er, mehr zu sich selbst. "`Das glaubst du doch wohl selbst nicht. Du bist es ja immernoch! Wie du da sitzt vor deinem Computer Tag und Nacht. Ich glaube, dass dir deine Einsamkeit nur deshalb missfällt, weil dir gesagt wurde, dass Einsamkeit schlecht ist. In Wirklichkeit bist du der einsamste Mensch der Welt - weil du einsam sein willst. Weil du keine Ahnung hast, welchen Wert ein Gegenüber, ein Freund haben kann."' Mit diesen Worten laufe ich aus dem Zimmer. Einen Moment noch warte ich in der Küche. Vielleicht wird er einsichtig, vielleicht ruft er mir hinterher. Aber nichts dergleichen geschieht, stattdessen höre ich schon nach kurzer Zeit wieder das vertraute klackern der Tastatur. Dann eben nicht. Mein PLan steht jedenfalls fest. Ich werde hinaus gehen, in die Welt, und er wird mich nicht mehr aufhalten können. Durch den Keller laufe ich in die Garage. Das Tor lässt sich ohne weiteres öffnen und hinter mir wieder schließen. Der leise Ton, denn das Tor macht, als es wieder auf dem Boden aufsetzt, ist wie ein Schlusspunkt für die Tage, die hinter mir liegen. Jona hat mich zwar gebaut - aber von nun an gehe ich selber meinen Weg.