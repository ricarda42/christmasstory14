Über Lichtermeere und unendliche Schwärze sind wir geflogen. Zuletzt war es lange dunkel gewesen, jetzt tauchen in der Ferne ein paar diffuse Lichter auf. Lautlos landet der Weihnachtsmann hinter einer gigantischen grauen Festungsanlage, die in unnatürliches Licht getaucht ist. Askaban. Vorsichtig schleichen wir näher.  Es dauert nicht lange, bis ich den Sicherungskasten der Flutlichter entdeckt habe. Für menschliche Augen ist er im Prinzip nicht auffindbar, aber mir ist es möglich, den schwachen Magnetismus, der von ihm ausgeht, zu detektieren. Ich laufe hin und mit ein paar elektrischen Signalen schalte ich die gesamte äußere Lichtanlage aus. Wir stehen in völliger Dunkelheit. Mir kommen allerdings mein Ultraschall und meine eingebaute Wärmebildkamera zu gute und so führe ich den Weihnachtsmann auf das Eingangstor zu. Ein paar ausgesendete Signale und ein ausgeklügelter Algorithmus zur Überwindung der Verschlüsselung reichen aus, um das Tor zu öffnen. Schnell finde ich eine Schnittstelle zur gesamten Alarmanlage und zur Stromversorgung des Komplexes. Hier sind die Barrieren allerdings etwas höher, meine Prozessoren laufen auf Hochtouren, probieren verschiedene Kombinationen, suchen nach Schwachstellen.

"`Eindringlinge!"', ruft jemand.

"`Schießt sie ab!"', ein anderer. "`Moment. Ist das der Weihnachtsmann?"'

"`Ho ho ho!"', sagt der Weihnachtsmann, "`Gibt es hier brave Wärter?"' Bei diesen Worten hält er einen Stapel Dollar-Noten hoch.

Die Wärter sind offensichtlich skeptisch. Dann hebt der eine sein Gewehr. "`Das ist Bestechung! Knallt sie ab!"'

Da habe ich es endlich geschafft - die Elektrik von Askaban liegt mir zu Füßen und augenblicklich gehen alle Lichter aus. Die Wärter brüllen sich irgendwas zu. Halb erwarte ich, dass als nächstes über all Taschenlampen angehen, aber diese Amateure haben augenscheinlich keine dabei. Trotzdem könnten sie auf die blöde Idee kommen, auf uns zu schießen, daher schalte ich das Laser-Schwert an und wende mich den Wärtern zu. Ich wollte sie nur ein wenig erschrecken, aber der Effekt ist größer als erwartet. Einen Roboter wie mich haben sie wohl noch nicht gesehen haben und schon gar nicht im Schein eines rotglühenden Laserschwerts. Einer von ihnen, wahrscheinlich der Wortführer, schießt trotzdem. Zum Glück hat sich der Weihnachtsmann längst in Deckung begeben. Für mich ist es natürlich kein Problem, der Kugel rechtzeitig auszuweichen. Als den Wärtern das klar wird, ergreifen sie die Flucht. Einer lässt sogar seine Waffe fallen, was für ein Idiot. Ich nehme sie an mich, für den Fall, dass mir noch Leute begegnen, die nicht ganz so dämlich sind. Über das Computersystem versuche ich Jona zu finden. Dann habe ich ihn endlich: Zelle 42a13. Mit einer knappen Wegbeschreibung und dem Laserschwert in der Hand schicke ich den Weihnachtsmann los. Ich selbst mache mich daran, ihm den Weg zu ebnen und sämtliche Sicherheitstore zu öffnen, die uns noch von Jona trennen. Fußgetrappel und laute Rufe erschallen über das Gelände - Hektik ist ausgebrochen, seitdem ich in den Systemen herumgepfuscht habe. Viel Personal scheint hier allerdings nicht zu sein - der Rest ist wohl im Weihnachtsurlaub, was für ein Pech. Trotzdem sollte ich mich beeilen. Als letztes öffne ich die Tür zu Jonas Zelle und mache mich schließlich selbst auf den Weg. Insgesamt begegne ich drei Wärtern, die aber sehr schnell keinen Ärger mehr machen, nachdem ich ihnen bei völliger Dunkelheit die Waffe aus der Hand geschossen habe und gleichzeitig gegen ihre Kugeln immun zu sein scheine. Ich hoffe nur, dass es der Weihnachtsmann ähnlich leicht hatte. Schon stehe ich vor der Zelle in der Jona gefangen gehalten wird. Auf einmal zögere ich: Was wird er wohl sagen? Schließlich habe ich ihm das alles eingebrockt.
