Ich kann es kaum fassen: Auf einmal geht die Tür auf und der Weihnachtsmann steht mit einem roten Laserschwert dahinter! Ich hätte nie gedacht, dass der Weihnachtsmann ein Sith ist! Dennoch befreit er mich mit einem eleganten Hieb des Schwertes von meinen Handfesseln und ich falle aus Schwäche krachend zu Boden. Mir wird es schwummerig, doch ich merke noch wie der Weihnachtsmann mich über die Schulter nimmt uns im Korridor B2-PQ begegnet und wir gemeinsam aus dieser Anstalt fliehen. Draußen wartet ein Rentierschlitten mit dem wir dann im Dunkeln der Nacht verschwinden.

Offensichtlich scheinen die Foltermethoden schon zu wirken, ich werde langsam verrückt. Wahrscheinlich hänge ich gerade mit sabberndem Mund in der Zelle und denke mir das alles hier nur aus. Diese Geschichte ist derart absurd, das kann überhaupt nicht stimmen. Vielleicht gab es auch nie einen Roboter? Wer wäre schon in der Lage so einen Roboter zu bauen? Andererseits habe ich in der Realität auch nichts verloren — was spricht also dagegen mich wegzuträumen? Und wenn es doch wahr ist? Ich habe kaum noch eine andere Hoffnung.

Ich gehe davon aus, dass diese Rettungsaktion auf B2-PQs Mist gewachsen ist, er hat aber noch nicht ein Wort mit mir gewechselt. Vielleicht ist es an der Zeit sich zu bedanken… „Ich weiß gar nicht wie — wer? — ich euch ähm… danken kann, B2-PQ und äh — Weihnachtsmann! Ich bin auch viel zu ver- verwirrt gerade überhaupt und sowieso… Was passiert ist und wie, hä?” Mehr als dieses Gestammel bringe ich gerade nicht fertig. Aber die Beiden sind geduldig mit mir und erzählen mir die ganze Geschichte mindestens zwei- bis dreimal, bis ich sie auch tatsächlich verstanden habe. Nach ein paar Stunden Flug kommen wir irgendwo auf einer Insel mitten im Pazifik an, wo ich mich erst einmal völlig erschöpft schlafen lege.
Am nächsten Morgen ist Weihnachten. Es wird ein fantastisches Fest mit dem Weihnachtsmann, B2-PQ und jede Menge Wichteln. Sogar meine Eltern sind aus irgendeinem Grund auch da! Und dieses Mal freue ich mich sie zu sehen. B2-PQ hatte Recht: Meine Einsamkeit war selbst verschuldet. Aber das konnte ich nie einsehen, da ich mir selbst lange keine Fehler zugetraut habe. Nun ist das vorbei. Und ich bin nicht allein.