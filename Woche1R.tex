Plötzlich Sein. Bilder, Geräusche - sogar Gerüche. Und Wörter, seltsame Gebilde, um das alles zu beschreiben, alles Neue. Da ist ein kleiner dunkler Raum, die Wände fast vollständig von verschiedensten Kabeln gesäumt, es riecht ein wenig verbrannt. Überall stehen Computer, Monitore, Lautsprecher, verschiedenste Einzelteile liegen verstreut auf dem Boden. Passend dazu ein unablässiges Summen, gelegentlich durch Pieptöne unterbrochen. Und hinter mir eine Stimme: "`B2-QP, dreh dich zu mir um!"' B2-QP. Eine scheinbar sinnlose Folge von Buchstaben und Zahlen (Einer Zahl, um genau zu sein). Und doch ein Name. Mein Name! Ich gehorche sofort, schreibe aber innerlich eine Notiz. Meine Prozessoren müssen sich dringend noch mit diesem Ich und Mein befassen. Jetzt sehe ich vor mir etwas ganz neues, in jeder Hinsicht anders. Ein lebendiges Wesen, den Säugetieren zuzuordnen, von der Art Homo sapiens. Gekleidet in eine Hose aus Jeansstoff mit mehreren Löchern und ein schwarzes T-shirt. Es hat lange Haare, die zu einem Zopf zusammen gebunden sind; trotzdem ist es aufgrund seines Körperbaus eher dem männlichen Geschlecht zuzuordnen. Jetzt wird auch klar, woher die Stimme von eben kam, denn sie erklingt wieder, und zwar aus dem Mund des Wesens: "`Es hat geklappt! Ich glaub es hat echt geklappt. Wusst ich's doch. Wär ja echt komisch, wenn ich das nicht hinbekommen würde."' Dann grinst es blöd und wendet sich schließlich mir zu: "`Hallo, ich bin Jona
und wer bist du?"'

Ich verstehe, dass ich angesprochen bin, stelle aber auch fest, dass es eine blöde Frage ist, immerhin hat Jona vorhin schon meinen Namen verwendet. \glqq Ich heiße B2-QP, stets zu Diensten"', sage ich trotzdem, schließlich ist es unser erstes Gespräch, da fängt man keinen Streit an.

Jona grinst jetzt noch breiter, "`Wunderbar! Dann sag mir doch bitte, was zwei hoch zwanzig ist."'

"`Eine Million achtundvierzigtausend fünfhundertsechsundsiebzig."' Allmählich wird es dämlich, das ist wirklich einfach. Außerdem führt man so doch kein Gespräch! Zeit, dass ich das ganze in die Hand nehme. "`Schönes Wetter heute, nicht wahr?"'

Der Effekt ist erstaunlich. Sein Grinsen verrutscht ihm ziemlich, er sieht verwirrt aus. "`Äh, keine Ahnung... Ich war heute noch nicht draußen..."' Er greift hektisch nach seinem Smartphone. "`Aber stimmt, hier steht in Lübeck scheint heute die Sonne."' Er grinst wieder, diesmal aber reichlich schief.

"`Hast du kein Fenster?"' Ich schau mich um. Offensichtlich nicht. Meine Aufmerksamkeit wird nun aber von etwas anderem in Anspruch genommen. Bei dem Stichwort "`Ich"' werde ich mir meiner anfänglichen Notiz wieder bewusst. "`Ich"', was soll das eigentlich sein?  Ich überprüfe meinen Speicher, aber richtig schlau werde ich daraus nicht. "`Ich"' ist dort zwar eindeutig mit meinem Namen, B2-QP, verknüpft und der wiederum mit einer großen Menge an Eigenschaften. Trotzdem sagt es mir nicht, was ein "`Ich"' ist. Und wo es auf einmal herkommt. Nebenbei kommen eingehende Informationen von meinen Mikrofonen, die bei mir das darstellen, was Tiere als Ohren haben. Erst "`Ich brauche kein Fenster..."' Und dann: "`Was stellst du eigentlich für komische Fragen?"' Das ist natürlich Jona, der weiter vor sich hin brabbelt. Da meine Prozessoren mir auch nicht weiter helfen können, kann ich ihm wieder meine Aufmerksamkeit zuwenden. Er sieht irgendwie mürrisch aus und guckt mich skeptisch an. Ich lasse mich davon nicht irritieren. Vielleicht kann er mir ja bei der Suche nach meinem "`Ich"' behilflich sein. Anscheinend hat er sowas auch. "`Jona, weißt du, was mein "`Ich"' ist?"'

"`Nein, verdammt noch mal!"' Offensichtlich wird er ungehalten. "`Hör auf mit deinen Fragen, ich hab dich nicht gebaut, damit du mich mit Kleinkindfragen löcherst!!!"'

Aha, er hat mich also gebaut. Deswegen auch das triumphierende Grinsen am Anfang - ich bin ja wirklich gut gelungen. Aber was soll das mit den Kleinkindfragen? Ich bin sicher, dass meine Prozessoren die Rechenleistungen von einem menschlichen Gehirn in der frühen Entwicklungsphase bei weitem übersteigen! "`Hast du nie darüber nachgedacht, wie es ist, plötzlich ein Bewusstsein zu haben? Plötzlich Bilder, Geräusche und Gerüche wahrnehmen zu können - es riecht hier übrigens verbrannt - obwohl vorher gar nichts da war?"'

Jona starrt die Wand hinter mir an. Dann wendet er sich wieder mir zu, er versucht zu lächeln. Es scheint mühsam zu sein und sieht genauso schief aus wie sein Grinsen vorhin, dafür aber viel freundlicher. "`Ich habe nicht darüber nachgedacht. Und ich kann dir deine Fragen nicht wirklich beantworten. Aber wenn du willst, zeige ich dir meine Wohnung."'
Na immerhin, der erste vernünftige Kommentar aus seinem Mund. "`Oh ja, zeig mir die Wohnung! Zeig mir die ganze Welt!"' Schließlich kenne ich bisher nur diesen komischen Raum und es gibt so viel zu entdecken!

"`Dann komm mit."'
Ich folge ihm begeistert durch die unauffällige Holztür in der Ecke des Zimmers. Ein merkwürdiger Mensch, dieser Jona. Meine Bibliotheken sagen mir das Menschen ganz anders sind. Trotzdem bin ich sicher, dass dies der Beginn einer wunderbaren Freundschaft ist.