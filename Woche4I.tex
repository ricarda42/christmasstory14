Das klingt doch nach einer Herausforderung: einen Roboter irgendwo auf der Welt finden, ohne die Wohnung zu verlassen. Als erstes simuliere ich was B2-PQ unter den gleichen Vorraussetzungen in meinem Weltmodell anstellt.

Na super, mein Programm spuckt aus, dass B2-PQ aus Neugierde schon längst das Land verlassen hat. Wohin ist leider völlig unklar, wobei Skandinavien und alle noch bestehenden Flugverbindungen von Lübeck aus wahrscheinlich sind. Allerdings ist ein Abflug aus Hamburg auch nicht unwahrscheinlich — also kann er praktisch überall sein. So komm ich nicht weiter. Zum Glück ist so ein Roboter relativ auffällig, mal die Nachrichten durchsuchen.

Kein Ergebnis. Aber dann gibt es ja noch soziale Netzwerke. Ich werde den restlichen Tag damit verbringen einen Bot %%%
zu schreiben, der sich durch Facebook, Instagram, Twitter, etc. durchfrisst %%% und nach Bilder mit B2-PQ sucht.

Der Bot hat tatsächlich Bilder gefünden: Bilder mit einer Gruppe junger Schweden in Malmö vor drei Tagen und ein Selfie mit einer Frau in London gestern! Auf Nachfrage hat sie mir auch zügig die Begegnung bestätigt. Aber danach verliert sich die Spur. Was kann man noch machen?

Ich habe noch eine Idee, aber die könnte mich in Probleme stürzen. London ist doch mit Überwachungskameras bespickt, wie andere Städte mit Tauben! Aber wie kommt man daran? Da heutzutage alles online ist und Menschen die Sicherheitssysteme erstellen — wie fast alle anderen Menschen auch — üblicherweise Fehler machen, ist es mir doch bestimmt möglich mich dort reinzuhacken!

33 Stunden und 4 Nervenzusammenbrüche später habe ich es endlich geschafft. Und ich habe ihn tatsächlich gefunden. Ich bin allerdings geraden dabei ein bisschen verrückt zu werden — auf dem Video sieht es so aus als würde ihn der Weihnachtsmann entführen würde! Der Weihnachtsmann! Das muss der Schlafentzug sein. Morgen sehen wir weiter…

Dummerweise ist es auch heute noch der Weihnachtsmann. Er steckt B2-PQ in einen Sack und fährt mit dem Taxi aus der Stadt heraus. Ich habe jetzt bei der Londoner Polizei noch ein bisschen „nachgeforscht“: Der Weihnachtsmann wird seit Jahren als Seriendieb gesucht — anscheinend klaut er immer Weihnachtsgeschenke in Kaufhäusern. Dann gibt es dort aber auch einen Verweis auf die CIA, die international gegen ihn ermittelt. Die CIA, klasse! Auf das wachsame Auge der Geheimdienste ist doch Verlass! Blöderweise haben die sich besser gegen Angriffe gewappnet als das arme Londoner Wachunternehmen. Alleine kriege ich das zwar nicht hin, aber glücklicherweise finden sich bestimmt jede Menge Aktivisten, die unter meiner Anleitung das Vorhaben angehen können.

War das ein Aufwand! Mit über 200 weiteren Programmierern habe ich es geschafft mir Zugriff auf die CIA Datenbank zu verschaffen. Und siehe da: Die CIA verfolgt den Weihnachtmann überhaupt nicht, sein Handeln wird geduldet. Und eine genaue Adresse gibt es auch — er wohnt in den Rocky Mountains. Rumms! Die Tür geht auf und ehe ich mich versehe werde ich brutal zu Boden geprügelt und in einen Wagen gezerrt. Dort gibt es noch mehr Schläge bis ich bewusstlos bin. Wer hat das kommen sehen?

Als ich aufwache bin ich in einem dunklen Raum im Stehen mit den Händen an die Decke gefesselt. Von überall schallt laute Musik. Ich bin wohl im Nachfolger des 2015 von Obama geschlossenen Guantanamo-Gefängnisses gelandet. Ich habe mir schon gedacht, dass sie einfach ein neues aufmachen würden.